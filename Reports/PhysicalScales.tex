\documentclass{jfm}

% Compilation
\usepackage{silence} % Silence latex compiler warnings
\WarningFilter{latex}{Command \@xhline has changed} % Filter out warning
  % caused by redefinition between jfm.cls class and array (loaded by
  % siunitx)

% Import custom style file containing common packages and options
\setlength{\paperheight}{\pdfpageheight} % JFM class removes paperheight definition and hyperref raises a warning

% Enable calculations inside tables
\usepackage{spreadtab}

% Import custom style file containing common packages and options
\usepackage{preamble}

\begin{document}

The physical dimensions of the solution are determined by two constants
(\eg $h$ and $g$) as well as $\epsilon$, $\mu$, and $P k_E/(\rho_w g)$,
or just $\epsilon$ and $P k_E/(\rho_w g)$ for solitary waves ($\mu = 6
\epsilon$).
We will therefore express all of the solitary wave's characteristic
dimension in terms of $h$, $g$, $\epsilon$, and $P k_E/(\rho_w g)$.

The ``width'' of the soliton is given by $\lambda_E = 2 \pi/k_E = 2 \pi
h /\sqrt{6 \epsilon}$.
The initial height is given by $H_0 = 2 \epsilon h$.
The characteristic speed is $c = \sqrt{gh}$.
The characteristic time is given by $t_C = 1/(\sqrt{gh}k_E) =
\sqrt{h/g}/\sqrt{6 \epsilon}$, and the ``wave-crossing time'' is $T_E =
\lambda_E/c = 2 \pi \sqrt{h/g}/\sqrt{6 \epsilon} = 2 \pi t_C$.
The characteristic energy growth time (\ie the time required for $E$ to
increase by a factor of $e$) is approximately
\begin{equation}
  \tau = \frac{1}{\gamma} =
  \pqty{\frac{P k_E}{\rho_w g}}^{-1} \frac{5}{c k_E}
  = \pqty{\frac{P k_E}{\epsilon \rho_w g}}^{-1} \frac{5}{\sqrt{6
  \epsilon^3}} \sqrt{\frac{h}{g}} \,.
\end{equation}
The simulation is run for a slow time period of $L_{t_1}' = 3$; this
corresponds to a full time of $L_t' = 3/\epsilon$, or a full, dimensional
time of
\begin{equation}
  L_t = \frac{3}{k_E \epsilon \sqrt{gh}}
  = \frac{3}{\sqrt{6\epsilon^3}} \sqrt{\frac{h}{g}} \,.
\end{equation}
During the simulation time $L_t$, the wave travels a distance of
approximately
\begin{equation}
  L_x = L_t \sqrt{gh} = \frac{3}{\sqrt{6\epsilon^3}} h \,.
\end{equation}
Finally, this corresponds to a wind speed of
\begin{equation}
  U_{\lambda/2} = U_{\pi h/\sqrt{6 \epsilon}}
  = \sqrt{gh}\pqty{1 +
    \sqrt{\frac{1}{5} \abs{\frac{P k_E}{\epsilon \rho_w g}}
    \frac{\rho_w}{\rho_a} \frac{1}{4.91 \sqrt{6 \epsilon}}}} \,.
\end{equation}

\begin{table}
\renewcommand\STprintnum[1]{\num[round-mode=figures, round-precision=2]{#1}}
\newcommand\gVal{9.8}
\newcommand\rhoaOnrhow{0.0011225}
\newcommand\PVal{0.25}
\begin{spreadtab}{{tabular}{c c | c c c c c c c c c}}
    @ $\epsilon$ & @ $h$ [\si{\meter}]
                 & @ $\lambda_E$ [\si{\meter}] & @ $H_0$ [\si{\meter}]
                 & @ $c$ [\si{\meter\per\second}]
                 & @ $T_E$ [\si{\second}] & @ $\tau$ [\si{\second}]
                 & @ U [\si{\meter\per\second}] & @ z [\si{\meter}]
                 & @ $L_t$ [\si{\second}]
                 & @ $L_x$ [\si{\meter}]
    \\
    0.1 & 10/pi*(6*[-1,0])^(0.5)
        & 2*pi*[-1,0]/(6*[-2,0])^(0.5) & 2*[-2,0]*[-3,0]
        & (\gVal*[-3,0])^(0.5)
        & 2*pi*([-4,0]/\gVal)^(0.5)/(6*[-5,0])^(0.5)
          & 1/\PVal*5/(6*[-6,0]^3)^(0.5)*([-5,0]/\gVal)^(0.5)
          & (\gVal*[-6,0])^(0.5)*(1+(1/5*\PVal/\rhoaOnrhow/4.91/(6*[-7,0])^(0.5))^(0.5))
          & [-6,0]/2
        & 3/(6*[-9,0]^3)^(0.5)*([-8,0]/\gVal)^(0.5)
        & [-1,0]*([-9,0]*\gVal)^(0.5)
    \\
    0.1 & 10
        & 2*pi*[-1,0]/(6*[-2,0])^(0.5) & 2*[-2,0]*[-3,0]
        & (\gVal*[-3,0])^(0.5)
        & 2*pi*([-4,0]/\gVal)^(0.5)/(6*[-5,0])^(0.5)
          & 1/\PVal*5/(6*[-6,0]^3)^(0.5)*([-5,0]/\gVal)^(0.5)
          & (\gVal*[-6,0])^(0.5)*(1+(1/5*\PVal/\rhoaOnrhow/4.91/(6*[-7,0])^(0.5))^(0.5))
          & [-6,0]/2
        & 3/(6*[-9,0]^3)^(0.5)*([-8,0]/\gVal)^(0.5)
        & [-1,0]*([-9,0]*\gVal)^(0.5)
    \\
    0.01 & 2.5
        & 2*pi*[-1,0]/(6*[-2,0])^(0.5) & 2*[-2,0]*[-3,0]
        & (\gVal*[-3,0])^(0.5)
        & 2*pi*([-4,0]/\gVal)^(0.5)/(6*[-5,0])^(0.5)
          & 1/\PVal*5/(6*[-6,0]^3)^(0.5)*([-5,0]/\gVal)^(0.5)
          & (\gVal*[-6,0])^(0.5)*(1+(1/5*\PVal/\rhoaOnrhow/4.91/(6*[-7,0])^(0.5))^(0.5))
          & [-6,0]/2
        & 3/(6*[-9,0]^3)^(0.5)*([-8,0]/\gVal)^(0.5)
        & [-1,0]*([-9,0]*\gVal)^(0.5)
    \\
    0.1 & 0.61
        & 2*pi*[-1,0]/(6*[-2,0])^(0.5) & 2*[-2,0]*[-3,0]
        & (\gVal*[-3,0])^(0.5)
        & 2*pi*([-4,0]/\gVal)^(0.5)/(6*[-5,0])^(0.5)
          & 1/\PVal*5/(6*[-6,0]^3)^(0.5)*([-5,0]/\gVal)^(0.5)
          & (\gVal*[-6,0])^(0.5)*(1+(1/5*\PVal/\rhoaOnrhow/4.91/(6*[-7,0])^(0.5))^(0.5))
          & [-6,0]/2
        & 3/(6*[-9,0]^3)^(0.5)*([-8,0]/\gVal)^(0.5)
        & [-1,0]*([-9,0]*\gVal)^(0.5)
\end{spreadtab}
  \caption{Solitary wave physical parameters as functions of $\epsilon$
    and $h$ for $P k_E/(\epsilon \rho_w g) = \PVal$, $\rho_a/\rho_w =
    \num{\rhoaOnrhow}$, and $g = \SI{\gVal}{\meter\per\second\squared}$.
  }
\undef\gVal
\undef\rhoaOnrhow
\undef\PVal
\end{table}

\end{document}
